\documentclass[12pt, letterpaper]{article}

\usepackage{amsmath}
\usepackage[margin=1in]{geometry}

% Preamble
\title{Preferences and Behaviour Choices}
\author{Ziteng Dong}
\date{\today}

\begin{document}

\maketitle

\newpage

\section{Introduction}
The individual decision problem starts with a set of \textbf{possible alternatives} ($X$) from which the agent must choose.\\

\textbf{Two approaches to modelling individual choice behaviour}

\begin{itemize}
    \item Preference-Based (Preference Relation)\\
    - imposing rationality axioms on agents' preference\\
    - analysing the consequences of the preference for agents' choice\\
    - it is about preferences
    \item Choice-Based\\
    - assumption on the choice behaviour directly\\
    - key assumption: Weak Axiom of Revealed Preference\\
    - modelling observable choice behaviour
\end{itemize}

\section{Preference Relation}
The preference relation is denoted by $\succeq$, which only tells the relation between two alternatives (binary relation) in the set of $X$.

\vspace{1em}

Say $x, y \in X$, 

    we read $x \succeq y$ as "x is at least as good as y" or "x is weakly preferred to y";
    
    we read $x \succ y$ as "x is (strictly) preferred to y";
    
    we read $x \sim y$ as "x is indifferent to y".

\subsection{Rationality}

\textbf{Definition}: A preference relation $\succeq$ is rational if it meets:
\begin{itemize}
    \item Completeness: for all $x, y \in X$, we have $x \succeq y$ or $y \succeq x$ or both;
    \item Transitivity: for all $x, y, z \in X$, if $x \succeq y$ and $y \succeq z$, then $x \succeq z$.
    
    Note that both are strong assumptions. 
    Completeness says that the agent has a clear preference between any two possible alternatives; 
    transitivity says that the agent will not have preference in a cycle when facing a sequence of pairwise options (i.e., a cycle such as $x \succeq y$, $y \succeq z$, then $z \succeq x$ will not happen).
\end{itemize}

Cases against transitivity:
\begin{itemize}
    \item \textit{Just perceptible differences}: When an agent is presented with pairwise options but with differences hard to discern (e.g., 98\% red and 96\% red), it might be indifferent. 
    If lighter colours are shown progressively, the agent may remain indifferent between options. However, if the agent is shown 98\% red and 88\% red, it might tell the difference.
    \item \textit{Framing problem}: Say Dr. Stephen wants to buy an IPhone (\pounds1200) and a compatible charger(\pounds25). 
    This is the price in Tesco and you tell him that Morrisons has a promotion on the charger with a deduction of \pounds 5. 
    Would he buy them at Tesco or Morrisons? What if the \%5 discount is on the IPhone?

    Experiments show that the fraction of participants who would go to Morrisons if the \pounds5 promotion is on the charger(\pounds25) is \textit{much higher} than the fraction if the \pounds5 promotion is on the IPhone (\pounds1200).
\end{itemize}

\subsection{Utility Functions}

A utility function quantifies the preference by assigning a numerical value to each element in $X$, ranking the elements to match consumer's preference.\\

\textbf{Definition}: $x(\cdot)$ is a utility function representing preference relations $\succeq$, if for $\forall \ x, y \in X$,

$$
x \succeq y \Leftrightarrow u(x) \geq u(y)
$$

Any\textit{ strictly increasing} function can be a utility function to represent a preference relation as long as it is rational. 
Or we can say that only a \textbf{rational} preference relation can be represented by a non-unique utility function. [See proof in Exercise 1]

\section{Choice Rules}

The agent's choice behaviour can be represent by the \textbf{choice structure}, which is formed by the \textbf{budget sets} $\mathcal{B}$ and \textbf{choice correspondence }$C(\cdot)$.\\ 

The budget sets $\mathcal{B} \subset 2^X$ and within $\mathcal{B}$, there are $B \in \mathcal{B}$ and $B \subset X$.\\

The choice correspondence $C(\cdot)$ takes $B$ as input and gives a set of commodities in B as output, or in mathematical form $C(B) \subset B$.\\

\textbf{Weak Axiom of Revealed Preference}\\

WARP assumes that consumer's observed choice behaviour should show a certain amount of consistency.\\

\textbf{Definition}: A choice structure satisfies WARP:\\

If for $B \in \mathcal{B}$ with $x, y \in B$, we have $x \in C(B)$, then for $\forall \ B^{\prime} \in \mathcal{B}$ with $x, y \in B^{\prime}$ and $y \in C(B^{\prime})$, we must also have $x \in C(B^{\prime})$.

This is saying that if we have $C(\{x, y\}) = x$, we cannot have $C(\{x, y, z\}) = y$.\\

\textit{Note}: $C(B)$ is the observed choice behaviour, and it only reveals what we observe.
We can use $\succeq^*$ to denote the observed choice behaviour. For example, say we have $C(\{x, y\}) = x$, then $x \succeq^* y$ or $x$ is \textit{revealed at least as good as} $y$.

\section{The Relationship between Preference Relations and Choice Rules}

\textbf{From $\succeq$ to $(\mathcal{B}, C(\cdot))$}.

If a $\succeq$ is rational, then the $(\mathcal{B}, C(\cdot))$ generated by it satisfies WARP.

\textbf{Proof}:     
If we know from the preference relations that $x \succeq y$, for $x, y \in B$, then we have $x \in C(B)$. 
Suppose $x, y \in B^{\prime}$, and $y \succeq z$ for $\forall \ z \in B^{\prime}$. The choice structure of this preference relation says $y \in C(B^{\prime})$. By transitivity, we have $x \succeq y$, so $y \in C(B^{\prime})$. Therefore, WARP is satisfied.\\

\noindent 
\textbf{From $(\mathcal{B}, C(\cdot))$ to $\succeq$}.

This way doesn't necessarily work. [See the counter example in the Exercises]

\section{Exercise}

1. Suppose the set of alternatives $X$ is finite. Show that a preference relation is rational if and only if it can be represented by a utility function.\\

\noindent
2.Suppose that $X=\{x, y, z\}$ and $\mathcal{B}$



\end{document}