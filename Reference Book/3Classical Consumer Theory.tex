\documentclass[12pt, letterpaper]{article}

\usepackage{amsmath}
\usepackage[margin=1in]{geometry}
\usepackage{graphicx}
\usepackage{float}
\usepackage{xcolor}


% Preamble
\title{Classical Consumer Theory}
\author{Ziteng Dong}
\date{\today}

\begin{document}

\maketitle

\newpage

\section{Introduction}

In this chapter, we study the preference-based approach to consumer demand.
\begin{itemize}
    \item Basic Properties of Preference Relations
        \begin{itemize}
            \item Monotonicity
        \end{itemize}
        \begin{itemize}
            \item Convexity
        \end{itemize}
    \item Continuity of Utility Function
    \item Utility Maximisation Problem
    \item Expenditure Minimisation Problem
    \item Duality
\end{itemize}

\section{Basic Properties of Preference Relations}

\textbf{Two assumption on PR}
\begin{itemize}
    \item \textbf{Monotonicity} 
    
    A PR is monotone if $x \in X$, $y \gg x$ implies $y \succ x$.
    Strictly monotone if $y > x$ implies $y \succ x$.

    \textbf{Local non-satiation} (A weaker form of monotonicity): A PR is locally non-satiated if $\forall \ x \in X$ and $\forall \ \epsilon$, $\exists \ y \in X$, s.t. $|y-x|<\epsilon$ and $y \succ x$.

\begin{figure}[H]
    \centering
    \includegraphics[width=0.75\linewidth]{Monotonicity.jpg}
    \caption{\textbf{Graphical Explanation of Monotonicity}.
            The right one should be \textit{local non-satiation}.}
    \label{Monotonicity}
\end{figure}

Local non-satiation rules out the \textit{thick} indifference set.
    
    \item \textbf{Convexity}
    
    A PR is convex if for $\forall \ x,y,z \in X$, $y \succeq x$ and $z \succeq x$, then $\alpha y + (1-\alpha) z \succeq x$, for $\alpha \in [0,1]$.

    Graphically, a PR is convex if its upper contour set is convex.

    \begin{figure}[H]
        \centering
        \includegraphics[width=0.75\linewidth]{Indifference Curve_Convex PR.jpg}
        \caption{\textbf{Illustration of Indifference Curve and Convex Preference}
        The left one gives examples of \textit{indifference curve}, which must be parallel.
        All the area to the right of an indifference curve is \textit{Upper Contour Set} and all the area to the left of an indifference curve is \textit{Lower Contour Set}.
        The local non-satiation assumption rules out the \textit{thick} indifference set.
        The middle one shows the difference between the \textit{convex} and \textit{concave} sets.
        The right one shows the difference between the \textit{convex} and \textit{strictly convex} sets.
        }
        \label{Indifference Curve_Convex PR}
    \end{figure}
    
\end{itemize}

\section{Continuity and Utility Function}

\textbf{Continuity} is another assumption imposed on a PR to ensure the existence of a utility function.\\

\textbf{Definition}: A PR is continuous if for any sequence of $\{(x_n, y_n)\}_{n=1}^{\infty}$, s.t. $y_n \succeq x_n$ for all $n$ and $(x, y) = \lim_{n \rightarrow \infty} (x_n, y_n)$, then $y \succeq x$.\\

Graphically, the upper and lower contour sets of a preference relation are closed.

\begin{figure}[H]
    \centering
    \includegraphics[width=0.75\linewidth]{UCS_LCS.jpg}
    \caption{\textbf{An Illustration of UCS and LCS}.
            If the \textcolor{red}{red dashed line} is an \textit{open} border of both sets, the this PR does not satisfy continuity.
            Because when $y$ in UCS approaches the \textcolor{red}{border}, it will jump to the LCS.
            However, if If the \textcolor{red}{red dashed line} is an \textit{closed} border of both sets, we can still have $y_U \succeq y_L$.
            }
    \label{UCS_LCS}
\end{figure}

\noindent
By now, we have the assumptions we need to have a utility function with nice properties to represent a PR.\\

\textbf{From PR to Utility Function}\\

\textbf{Proposition}: If a PR is rational and continuous, then there exists a continuous utility function $u(x)$ that represents this PR.

\textit{Note}: Not all $u(x)$ representing a PR are continuous.\\

\noindent
With a utility function, we can do a lot of things in a numerical way.

\begin{itemize}
    \item Indifferent set can be written as $\{y \in X|y \sim x\} \Leftrightarrow \{y \in X|u(y) = u(x)\}$.
    \item By taking derivative of $u(x)$ w.r.t. two goods:
        \begin{align*}
        \underbrace{\frac{\partial \ u}{\partial \ x_1} dx_1}_{Marginal \ Change \ in \ x_1} 
        + \underbrace{\frac{\partial \ u}{\partial \ x_2} dx_2}_{Marginal \ Change \ in \ x_2}
        =   \underbrace{0}_{Still \ on \ Indifference \ Curve}
        \end{align*}

    Then the slope of the indifference curve is:

        \begin{align*}
        \frac{dx_2}{dx_1} = - \underbrace{\frac{\partial \ u / \partial \ x_1}{\partial \ u / \partial \ x_2}}_{Marginal \ Rate \ of \ Substitution}
        \end{align*}
\end{itemize}

\textbf{Implications of Properties of PR on Utility Function}\\
\begin{itemize}
    \item Monotonicity of PR implies an increasing $u(x)$ :
        \begin{align*}
        \frac{\partial \ u}{\partial \ x_1} \geq 0
        \qquad
        \frac{\partial \ u}{\partial \ x_2} \geq 0
        \end{align*}
        more consumption has larger utility.
    \item Convexity of PR implies a quasi-concave $u(x)$:
        $u(\alpha x +(1-\alpha)y) \geq min\{u(x), u(y)\}$ for $\alpha \in [0,1]$.

        A quasi-concave $u(x)$ means a diminishing Marginal Rate of Return (second derivatives are negative).
\end{itemize}




Utility Maximisation Problem, Expenditure Minimisation Problem, and their Duality will be covered in the next Note.

\end{document}