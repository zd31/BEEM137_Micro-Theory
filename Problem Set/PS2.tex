\documentclass{article}

\usepackage{amsmath}

\title{Problem Set 2 \\ \large BEEM137 Micro Theory}
\author{Ziteng Dong}
\date{\today}

\begin{document}

\maketitle

\section{}

Let $p_1, p_2$ be the prices of goods in the first and the second period;

\noindent Let $x_1, x_2$ be the amount of goods consumed by the consumer in the first and the second period.

Then we have the Walrasian Budget:
$$\{x_1, x_2 | p_1 x_1 + p_2 x_2 \leq w \}$$

\section{}

The homogeneity of degree zero implies that 
$$x(p, w) = x(\lambda p, \lambda w), for \ \lambda>0$$

By multiplying $\lambda$ by wealth $w$ and price of each commodity $p$, we have 
$$
x_1 (\lambda p, \lambda w) 
= 
    \frac{\lambda p_2}{\lambda p_1 + \lambda p_2 + \lambda p_3} \frac{\lambda w}{\lambda p_1} 
= 
    \frac{p_2}{p_1 + p_2 + p_3} \frac{w}{p_1} = x_1 (p, w)
$$ 

The cancellation occurs to every demand function, irrelevant to the value of $\beta$.

\vspace{2em}

The Walras' law indicates that 
$$ p \cdot x (p, w) = w$$

To satisfy Walras' Law, the following equation must be true:

\begin{align*}
p_1 \cdot x_1 (p, w) + 
p_2 \cdot x_2 (p, w) + 
p_3 \cdot x_3(p, w) 
&= 
    \frac{p_1 p_2}{p_1 + p_2 + p_3} \frac{w}{p_1} + 
    \frac{p_3 p_2}{p_1 + p_2 + p_3} \frac{w}{p_2} + 
    \frac{\beta p_1 p_3}{p_1 + p_2 + p_3} \frac{w}{p_3} \\ 
&= 
    \frac{\beta p_1 + p_2 + p_3}{p_1 + p_2 + p_3} \ w \\
&=
    w
\end{align*}

If $\beta = 1$, then the above equation is true; 
otherwise, it cannot hold. 

\vspace{2em}

Therefore, when $\beta=1$ the demand function satisfies both hod 0 and Walras' Law;
when $\beta \in (0,1)$, it only satisfies hod 0.

\section{}

\subsection{}

The Walras' law indicates that 
$$ p \cdot x (p, w) = w$$

In this case where $p= \{3, 6\}$, we have 
$$
p \cdot x (p, w) 
=
    3 \times 4 + 6 \times 5 
= 
    42
$$

So, the set of possible demands at the new prices should be
$$W = \{(x_1, x_2) | 5 x_1 + 2 x_2 = 42 \}$$

\subsection{}

The cost of the original bundle at the new price is:
$$
p^{\prime} \cdot x (p, w) 
=
    5 \times 4 + 2 \times 5 
= 
    30
$$
which is lower than the wealth $w=42$.
This is saying that the original bundle is still affordable at the new price.

To satisfy WARP, the set of possible demand should be
$$
\{(x_1, x_2) | 
    p^{\prime} x(p^{\prime},w) \leq 42  
    \ \ and \ \ 
    px(p^{\prime},w) > 42\}
$$

\section{}

The matrix of costs of each observed choice and the prices:

\[
\left[
\begin{array}{c|ccc}
  & p^1 & p^2 & p^3 \\ \hline
x^1 & 8 & 8 & 9 \\
x^2 & 9 & 8 & 8 \\
x^3 & 8 & 9 & 8
\end{array}
\right]
\]

At price $p^1$, $x^2$ costs 9 wealth which is not affordable. 
We have observed that $x^1$ is revealed at least as good as $x^3$;

At price $p^2$, $x^3$ costs 9 wealth which is not affordable. 
We have observed that $x^2$ is revealed at least as good as $x^1$;

At price $p^3$, $x^1$ costs 9 wealth which is not affordable. 
We have observed that $x^3$ is revealed at least as good as $x^2$.

These revealed choices satisfy WARP and the property of completeness of rational preference relations.
However, they are not transitive, as $x^1 \succ x^3$, $x^2 \succ x^1$, and $x^3 \succ x^2$, which implies $x^2 \succ x^1 \succ x^3 \succ x^2$.

\end{document}