\documentclass[12pt, a4paper]{article}

% preamble
\title{Problem Set 1 \\ \large BEEM137 Micro Theory}
\author{Ziteng Dong}
\date{\today}

\begin{document}
\maketitle

\section{}

\subsection{Method 1}

\begin{center}
    \begin{tabular}{ccc}
      $|X|$ & $|2^X|$ &      \\
        0   &   1     & $2^0$\\
        1   &   2     & $2^1$\\
        2   &   4     & $2^2$\\
        ... &  ...    & ...  \\
        n   &         & $2^n$\\
    \end{tabular}
\end{center}

\subsection{Method 2}

Each element in the set $X$ is either in its subset or not in its subset.
In other words, here are two possibilities for each item.
So, the total number of possible subsets of $X$ with $n$ elements is $2 \times 2 \times ... \times 2 = 2^n$.

\subsection{Method 3}

???

\section{}

NO.
This choice correspondence does not satisfy WARP.

$C(\{x, y\}) = \{x, y\} \Rightarrow x \sim^* y$
Or in words, the agent has revealed indifference between $x$ and $y$ when they are presented in the budget set.

According to WARP, the agent should choose both $x$ and $y$ when they are in the budget set and one of the two is chosen.
However, the second correspondence $C(\{x, y, z\}) = \{x, z\}$ violates this rule.

\section{}

$x \succ y$ implies that $x \succeq y$ and $\neg y \succeq x$.

Since $\succeq$ is rational, it has the property of transitivity. 
Based on this, we have $x \succeq y \succeq z \Rightarrow x \succeq z$.

We can also have $z \succeq y$ given $y \succeq z$. However, the relation $z \succeq x$ cannot be established.

\section{}

Since the correspondence $C$ is single-valued, there must be only one output given one input $B$.

$C(B)=x$ reveals that only $x$ will be chosen when $x$ and $y$ are in the same budget set. 
In other words, if $y$ were chosen, then $x$ is not in the same budget set.
 
Therefore, $C(B^{\prime}) = y$ with $x, y \in B^{\prime}$ is contrary to the rule above.

\section{}

If a preference relation is rational, then it has properties of completeness and transitivity.

A utility function $u(.)$ has following properties:

\begin{itemize}

    \item $\forall x, y \in X$, we have either $u(x) \geq u(y)$ or $u(y) \geq u(x)$, which proves completeness;

    \item Because $u(.)$ is strictly increasing, if $u(x) \geq u(y)$ and $u(y) \geq u(z)$, then we have $u(x) \geq u(z)$, which establishes transitivity.
    
\end{itemize}

Therefore, a rational preference relation can be represented by a utility function.

\section{}

If a preference relation is rational, then it must have the properties of completeness and transitivity.
To satisfy completeness, every two choices must be comparable.

Let $X = {x, y, z} $, $\mathcal{B}={{x, y, z}}$, and $B = {x, y ,z}$.
If we have $C(B)=x$, we know that x is revealed to be at least as good as $y$ and $z$, which essentially satisfies WARP. However, we don't know if $y$ and $z$ are comparable. So, it fails to establish completeness.


\end{document}